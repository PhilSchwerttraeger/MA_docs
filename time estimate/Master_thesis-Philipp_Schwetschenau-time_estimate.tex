% Diese Vorlage wurde mit freundlicher Unterstützung von Felix Neumann angefertigt.

\documentclass[12pt,a4paper]{article}

\usepackage[USenglish]{babel}    % Deutsche Sprache in automatisch generiertem
\usepackage{latexsym}         % Fuer recht seltene Zeichen
\usepackage[utf8]{inputenc}   % =E4 =F6 =FC =DF; danach  geht auch das ß richtig
\usepackage{caption}          % Figure-Captions formatieren
\usepackage{sectsty}          % Section headings formatieren
\usepackage[fixlanguage]{babelbib}
\usepackage[a4paper,lmargin={2.5cm},rmargin={2.5cm},tmargin={3cm},bmargin={2.5cm}]{geometry}
\usepackage[sort&compress,square,comma,authoryear]{natbib}
\usepackage{booktabs}
\usepackage{tabularx}
\usepackage{xcolor}
\usepackage{enumitem}

\pdfinfo{
	/Title		(Graphical Specification Language for the Entity-Labeling Aspect)
	/Subject	(Exposé -- Master Thesis)
	/Author		(Philipp Schwetschenau)
}

\selectbiblanguage{USenglish}
\allsectionsfont{\sffamily}
\captionsetup{margin=1cm,font=small,labelfont=bf}

\newcommand\notice[1]{}
\newcommand\seppar{ \vspace{2ex} \noindent }
\renewcommand*\theenumii{\labelenumi\arabic{enumii}}
\renewcommand*\labelenumii{\theenumii}

\newenvironment{redtext}{\color{gray} \it{}}{\ignorespacesafterend}


\begin{document}

\title{{\bf Graphical Specification Language \\for the Entity-Labeling Aspect} \\ 
\begin{large}Structure and time estimate -- Master Thesis\end{large}
}
\author{
	Philipp Schwetschenau \\
	Technische Universität Ilmenau \\
	philipp.schwetschenau@tu-ilmenau.de
}
\date{\today}

\maketitle

\section{Time estimate} 
\begin{table}[h]
   \begin{tabularx}{\textwidth}{p{1cm} X p{5cm}} \toprule
   1 	& Introduction 							& 1w 	\\ 
   2 	& Fundamentals 							& 3w	\\
   3 	& Design: Graphical specification language 	& 6w	\\
   4 	& Design: Editor GUI 						& 2w	\\
   5 	& Implementation 						& 5w	\\
   6 	& Evaluation 								& 3w	\\
   7 	& Outlook and Future Work 				& 1w	\\
   8 	& Summary 								& 1w	\\	
   	& Total									& 22w 	\\ \bottomrule
    \\ 
  \end{tabularx}
  \caption{Estimated working time in weeks}
 \end{table}
  
\cleardoublepage

\section{Content structure}

   \begin{enumerate}[noitemsep]
      \item Introduction
  \begin{enumerate}[noitemsep]
         \item Domain
         \item Motivation
         \item Goal
         \item Approach
      \end{enumerate}
      \item Fundamentals
  \begin{enumerate}[noitemsep]
         \item Security Models
         \item Aspect-oriented Security Engineering and Entity-Labeling Aspect
         \item Graphical notation models \begin{redtext}(UML, ERD, RBAC-notation by Sandhu)\end{redtext}
         \item Gestalt laws and human optical perception
         \item GUI design \begin{redtext}(design patterns, usability)\end{redtext}
      \end{enumerate}
      \item Design: Graphical specification language
  \begin{enumerate}[noitemsep]
         \item Concept \begin{redtext}(approach, basic ideas, adoptions from literature)\end{redtext}
         \item Elements
         \item Relationships
         \item Structure
	  \item \begin{redtext}Optional: Visualization on the higher abstraction level\end{redtext}
      \end{enumerate}
      \item Design: Editor GUI
  \begin{enumerate}[noitemsep]
         \item Structure
         \item Sections
      \end{enumerate}
      \item Implementation
  \begin{enumerate}[noitemsep]
         \item Implementation base \begin{redtext}(Qt, MVC)\end{redtext}
         \item Structure
         \item GUI sections
      \end{enumerate}
      \item Evaluation
  \begin{enumerate}[noitemsep]
         \item Graphical specification language
         \item Editor
      \end{enumerate}
      \item Outlook and Future Work
	\item Summary
   \end{enumerate}
   
\end{document}
