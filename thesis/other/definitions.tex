% General Setup
\usepackage{ifthen}		% If-Then-Statements
\usepackage{pdfpages}
\usepackage{hyperref}		% Hyperlinks & PDF specific information
\hypersetup{
	pdftitle={Master thesis - Philipp Schwetschenau},
	pdfsubject={Master thesis},
	pdfauthor={Philipp Schwetschenau},
	pdfborder={0 0 0}
}
\usepackage[nounderscore]{syntax}

% Language & Encoding
\usepackage[T1]{fontenc}
\usepackage[USenglish]{babel}
\usepackage[fixlanguage]{babelbib}
\usepackage[utf8]{inputenc}
\usepackage{blindtext}
\usepackage{caption}


% Page Geometry
\usepackage[a4paper]{geometry}
\geometry{
	twoside,
	top=3cm,
	bottom=3cm
}

% Literature
\usepackage{url}
\usepackage[sort&compress,square,comma,authoryear]{natbib}

% Fonts & Symbols
\usepackage{latexsym}		% Rare symbols
\usepackage{amsfonts}		% Math fonts
\usepackage{amssymb}		% Symbols
\usepackage{amsmath}		% Symbols
\usepackage{lmodern}		% "Latin Modern" ("Computer Modern"++)

\newcommand{\origttfamily}{}	% Separator for typewriter
\let\origttfamily=\ttfamily
\renewcommand{\ttfamily}{\origttfamily \hyphenchar\font=`\-}


%
% Document Layout

\usepackage[activate]{pdfcprot}	% margin kerning

% Header & Footer
\usepackage{fancyhdr}		% format headers
\pagestyle{fancy}
\fancyhf{}
\setlength{\headheight}{15pt}
\fancyhead[LE,RO]{\sffamily \thepage}
\fancyhead[RE]{\sffamily \nouppercase{\leftmark}}
\fancyhead[LO]{\sffamily \nouppercase{\rightmark}}
\renewcommand{\headrulewidth}{0.4pt}
\fancypagestyle{plain}{
	\fancyhead[RE,LO]{}
	\renewcommand{\headrulewidth}{0pt}
}
\fancypagestyle{simple}{
	\fancyhead[RE,LO]{}
	\renewcommand{\headrulewidth}{0pt}
}
\fancypagestyle{light}{
	\fancyhead[RE,LO]{}
}

% ClearDoublePage fix
\makeatletter 
\def\cleardoublepage{\clearpage\if@twoside \ifodd\c@page\else% 
\hbox{}% 
\thispagestyle{simple}
\newpage% 
\if@twocolumn\hbox{}\newpage\fi\fi\fi}
\makeatother 

% Headlines
\usepackage{titlesec}
\setcounter{secnumdepth}{3}
\titleformat{\chapter}[display]%
	{\huge\center\bf}%
	{\large\mdseries CHAPTER \thechapter}%
	{0cm}{}[\vspace{2ex}\titlerule]
\titlespacing*{\chapter}{0pt}{0ex}{8ex}
\titleformat{\subsubsection}{\normalsize\bfseries}{\thesubsubsection}{.75em}{}
\titleformat{\paragraph}[runin]{\bfseries}{}{0pt}{}[.]
\titleformat{\subparagraph}[runin]{\itshape}{}{0pt}{}[.]

% Table of Contents
\usepackage[titles]{tocloft}

\setlength{\cftbeforetoctitleskip}{0ex}
\setlength{\cftaftertoctitleskip}{0ex}
\renewcommand{\cfttoctitlefont}{}

\setlength{\cftbeforeloftitleskip}{4ex}
\setlength{\cftafterloftitleskip}{1ex}
\renewcommand{\cftloftitlefont}{\LARGE}

\setlength{\cftbeforelottitleskip}{4ex}
\setlength{\cftafterlottitleskip}{1ex}
\renewcommand{\cftlottitlefont}{\LARGE}

\newcommand\listingname{Verzeichnis der Listings}
\newlistof[chapter]{listing}{lst}{\listingname}
\setlength{\cftbeforelsttitleskip}{4ex}
\setlength{\cftafterlsttitleskip}{1ex}
\renewcommand{\cftlsttitlefont}{\LARGE}

\newcommand\theoremsname{Theoremverzeichnis}
\newlistof[chapter]{theorems}{lthm}{\theoremsname}
\setlength{\cftbeforelthmtitleskip}{4ex}
\setlength{\cftafterlthmtitleskip}{1ex}
\renewcommand{\cftlthmtitlefont}{\LARGE}

\setcounter{tocdepth}{2}
\setlength{\cftbeforechapskip}{1.0ex}
\setlength{\cftbeforesecskip}{0ex}
\setlength{\cftbeforesubsecskip}{-.2ex}
\newcommand\tocentry[1]{\addcontentsline{toc}{chapter}{#1}}
\newcommand{\ttsubsection}[1]{\subsection[\texorpdfstring{\texttt{\slshape #1}}{#1}]{\texttt{#1}}}
\newcommand\addtotheorems[2]{
	\refstepcounter{theorems}
	\addcontentsline{lthm}{theorems}{\protect\numberline{\thetheorems}\textbf{#1:} #2}
}
\newcommand\addlistspace[1]{
	\addtocontents{#1}{\vspace{1.3ex}}
}

% Glossar
\usepackage[number=none,style=altlist]{glossary}
\renewcommand{\glosslabel}[2]{\sffamily #2}
\makeglossary


%
% Page Elements

% Captions & Figures
\usepackage{graphicx}		% include graphics
\graphicspath{{./figures/}}

% Tables
\usepackage{booktabs}

% Theorems
\usepackage{framed}		% Frames for theorems.
\usepackage[framed,thmmarks,amsmath]{ntheorem}	% extended theorem enviroments
\usepackage{shadethm}		% theorems with colored background
\theoremheaderfont{\sffamily\bfseries}
\theorembodyfont{}
\theoremstyle{break}
\theoremseparator{.}
\theoremindent0cm
\theoremsymbol{}

\newshadetheorem{xtheorem}{Theorem}[chapter]
\newshadetheorem{xlemma}[xtheorem]{Lemma}
\newshadetheorem{xdefinition}[xtheorem]{Definition}
\newshadetheorem{xrequirement}[xtheorem]{Requirement}

\newenvironment{theorem}[1][]{%
	\addtotheorems{Theorem}{#1}
	\begin{xtheorem}[#1]%
}{\end{xtheorem}}

\newenvironment{lemma}[1][]{%
	\addtotheorems{Lemma}{#1}
	\begin{xlemma}[#1]%
}{\end{xlemma}}

\newenvironment{definition}[1][]{%
	\addtotheorems{Definition}{#1}
	\begin{xdefinition}[#1]%
}{\end{xdefinition}}

\newenvironment{requirement}[1][]{%
	\addtotheorems{Requirement}{#1}
	\begin{xrequirement}[#1]%
}{\end{xrequirement}}

\newshadetheorem{xapxtheorem}{Theorem}[section]
\newenvironment{apxtheorem}[1][]{%
	\addtotheorems{Theorem}{#1}
	\begin{xapxtheorem}[#1]%
}{\end{xapxtheorem}}

\theoremstyle{nonumberplain}
\theoremseparator{.}
\theoremheaderfont{\sffamily\itshape}
\theoremsymbol{\ensuremath{\Box}}
\newtheorem{proof}{Proof}
\newtheorem{apxproof}{Proof}

% Listings
\usepackage{listings}
\lstset{
	basewidth={0.5em,0.45em},
	frame=lines,
	framerule=\lightrulewidth,
	captionpos=b,
	lineskip=-1pt,
%	float=hbt,
	xleftmargin=1cm,
	xrightmargin=1cm,
	aboveskip=0.5cm,
	belowskip=0.5cm,
}
\lstnewenvironment{java}[2]{
	\refstepcounter{listing}
	\addcontentsline{lst}{listing}{\protect\numberline{\thelisting}#1}
	\lstset{
		language=C++,
		basicstyle=\ttfamily,
		commentstyle=\sffamily,
		lineskip=-2pt,
		tabsize=4,
		#2
	}
}{}
\lstdefinelanguage{pseudocode}{
	sensitive=true,
	alsodigit={:},
	morekeywords={
		Algorithmus,Eingabe:,Ausgabe:,Variablen:,%
		when,if,then,else,end,atomic,out,%
		case,is,repeat,while,do,until},
}
\lstdefinelanguage[distributed]{pseudocode}
	[]{pseudocode}{
	alsodigit={:},
	morekeywords={Async,Sync,%
		Constants:,Variables:,Input:,Action:,Action,%
		send,to,from},
	deletekeywords={Algorithmus,Eingabe:,Ausgabe:,Variablen:}
}
\newcommand{\setpseudocode}[2]{
	\lstset{
		% Syntax
		language=[#1]{pseudocode},
		mathescape=true,
		escapechar=\#,
		tabsize=4,
%		gobble=4,
		literate={:=}{{$\gets$}{\:}}2 {->}{{$\rightarrow$}{\:}}2,
		% Style
		basicstyle=\rmfamily,
		commentstyle=\sffamily,
		moredelim=[is][\itshape]{//}{//},
		moredelim=[il][\sffamily]{**},
		% Layout & Placement
		numbers=left,
		numberstyle=\tiny,
		columns=flexible,
		breaklines=true,
		breakatwhitespace=true,
		breakindent=2em,
		% User defined
		#2
	}
}
\lstnewenvironment{distalg}[2]{%
	\refstepcounter{listing}
	\addcontentsline{lst}{listing}{\protect\numberline{\thelisting}#1}
	\setpseudocode{distributed}{#2}
}{}
\lstnewenvironment{pseudocode}[2]{%
	\refstepcounter{listing}
	\addcontentsline{lst}{listing}{\protect\numberline{\thelisting}#1}
	\setpseudocode{}{#2}
}{}

\newcommand\textcall[1]{\textsc{#1}}
\newcommand\call[2]{\ensuremath{\operatorname{\textcall{#1}}(#2)}}
\newcommand\callname[2]{\ensuremath{\operatorname{#1}(#2)}}
\newcommand\topin[1]{\ensuremath{\operatorname{In}({#1})}}
\newcommand\topout[1]{\ensuremath{\operatorname{Out}({#1})}}


%
% Inline
\usepackage{url}		% Cite URLs
\usepackage{numprint}		% Format numbers
\newcommand\notice[1]{}		% Notice
\newcommand\seppar{ \vspace{2ex} \noindent } % New paragraph
\newcommand\name[1]{{\em #1}} 	% Names


\newenvironment{BNF}
  {\captionsetup{type=lstlisting}}
  {}
  
\renewcommand{\labelitemi}{$\bullet$}
\renewcommand{\labelitemii}{$\bullet$}
\renewcommand{\labelitemiii}{$\bullet$}
\renewcommand{\labelitemiv}{$\bullet$}

\usepackage[inline]{enumitem}


\usepackage{xcolor}
\usepackage[tikz]{bclogo}
\usepackage[framemethod=tikz]{mdframed}
\usepackage{lipsum}
\usepackage[many]{tcolorbox}

\definecolor{bgblue}{RGB}{245,243,253}
\definecolor{ttblue}{RGB}{91,194,224}

\mdfdefinestyle{mystyle}{%
  rightline=true,
  innerleftmargin=10,
  innerrightmargin=10,
  outerlinewidth=1pt,
  topline=false,
  rightline=true,
  bottomline=false,
  skipabove=\topsep,
  skipbelow=\topsep
}