\documentclass[twoside, openright, 12pt]{book}
\sloppy

% Preamble
% General Setup
\usepackage{ifthen}		% If-Then-Statements
\usepackage{pdfpages}
\usepackage{hyperref}		% Hyperlinks & PDF specific information
\hypersetup{
	pdftitle={Master thesis - Philipp Schwetschenau},
	pdfsubject={Master thesis},
	pdfauthor={Philipp Schwetschenau},
	pdfborder={0 0 0}
}
\usepackage[nounderscore]{syntax}

% Language & Encoding
\usepackage[T1]{fontenc}
\usepackage[USenglish]{babel}
\usepackage[fixlanguage]{babelbib}
\usepackage[utf8]{inputenc}
\usepackage{blindtext}
\usepackage{caption}


% Page Geometry
\usepackage[a4paper]{geometry}
\geometry{
	twoside,
	top=3cm,
	bottom=3cm
}

% Literature
\usepackage{url}
\usepackage[sort&compress,square,comma,authoryear]{natbib}

% Fonts & Symbols
\usepackage{latexsym}		% Rare symbols
\usepackage{amsfonts}		% Math fonts
\usepackage{amssymb}		% Symbols
\usepackage{amsmath}		% Symbols
\usepackage{lmodern}		% "Latin Modern" ("Computer Modern"++)

\newcommand{\origttfamily}{}	% Separator for typewriter
\let\origttfamily=\ttfamily
\renewcommand{\ttfamily}{\origttfamily \hyphenchar\font=`\-}


%
% Document Layout

\usepackage[activate]{pdfcprot}	% margin kerning

% Header & Footer
\usepackage{fancyhdr}		% format headers
\pagestyle{fancy}
\fancyhf{}
\setlength{\headheight}{15pt}
\fancyhead[LE,RO]{\sffamily \thepage}
\fancyhead[RE]{\sffamily \nouppercase{\leftmark}}
\fancyhead[LO]{\sffamily \nouppercase{\rightmark}}
\renewcommand{\headrulewidth}{0.4pt}
\fancypagestyle{plain}{
	\fancyhead[RE,LO]{}
	\renewcommand{\headrulewidth}{0pt}
}
\fancypagestyle{simple}{
	\fancyhead[RE,LO]{}
	\renewcommand{\headrulewidth}{0pt}
}
\fancypagestyle{light}{
	\fancyhead[RE,LO]{}
}

% ClearDoublePage fix
\makeatletter 
\def\cleardoublepage{\clearpage\if@twoside \ifodd\c@page\else% 
\hbox{}% 
\thispagestyle{simple}
\newpage% 
\if@twocolumn\hbox{}\newpage\fi\fi\fi}
\makeatother 

% Headlines
\usepackage{titlesec}
\setcounter{secnumdepth}{3}
\titleformat{\chapter}[display]%
	{\huge\center\bf}%
	{\large\mdseries CHAPTER \thechapter}%
	{0cm}{}[\vspace{2ex}\titlerule]
\titlespacing*{\chapter}{0pt}{0ex}{8ex}
\titleformat{\subsubsection}{\normalsize\bfseries}{\thesubsubsection}{.75em}{}
\titleformat{\paragraph}[runin]{\bfseries}{}{0pt}{}[.]
\titleformat{\subparagraph}[runin]{\itshape}{}{0pt}{}[.]

% Table of Contents
\usepackage[titles]{tocloft}

\setlength{\cftbeforetoctitleskip}{0ex}
\setlength{\cftaftertoctitleskip}{0ex}
\renewcommand{\cfttoctitlefont}{}

\setlength{\cftbeforeloftitleskip}{4ex}
\setlength{\cftafterloftitleskip}{1ex}
\renewcommand{\cftloftitlefont}{\LARGE}

\setlength{\cftbeforelottitleskip}{4ex}
\setlength{\cftafterlottitleskip}{1ex}
\renewcommand{\cftlottitlefont}{\LARGE}

\newcommand\listingname{Verzeichnis der Listings}
\newlistof[chapter]{listing}{lst}{\listingname}
\setlength{\cftbeforelsttitleskip}{4ex}
\setlength{\cftafterlsttitleskip}{1ex}
\renewcommand{\cftlsttitlefont}{\LARGE}

\newcommand\theoremsname{Theoremverzeichnis}
\newlistof[chapter]{theorems}{lthm}{\theoremsname}
\setlength{\cftbeforelthmtitleskip}{4ex}
\setlength{\cftafterlthmtitleskip}{1ex}
\renewcommand{\cftlthmtitlefont}{\LARGE}

\setcounter{tocdepth}{2}
\setlength{\cftbeforechapskip}{1.0ex}
\setlength{\cftbeforesecskip}{0ex}
\setlength{\cftbeforesubsecskip}{-.2ex}
\newcommand\tocentry[1]{\addcontentsline{toc}{chapter}{#1}}
\newcommand{\ttsubsection}[1]{\subsection[\texorpdfstring{\texttt{\slshape #1}}{#1}]{\texttt{#1}}}
\newcommand\addtotheorems[2]{
	\refstepcounter{theorems}
	\addcontentsline{lthm}{theorems}{\protect\numberline{\thetheorems}\textbf{#1:} #2}
}
\newcommand\addlistspace[1]{
	\addtocontents{#1}{\vspace{1.3ex}}
}

% Glossar
\usepackage[number=none,style=altlist]{glossary}
\renewcommand{\glosslabel}[2]{\sffamily #2}
\makeglossary


%
% Page Elements

% Captions & Figures
\usepackage{graphicx}		% include graphics
\graphicspath{{./figures/}}

% Tables
\usepackage{booktabs}

% Theorems
\usepackage{framed}		% Frames for theorems.
\usepackage[framed,thmmarks,amsmath]{ntheorem}	% extended theorem enviroments
\usepackage{shadethm}		% theorems with colored background
\theoremheaderfont{\sffamily\bfseries}
\theorembodyfont{}
\theoremstyle{break}
\theoremseparator{.}
\theoremindent0cm
\theoremsymbol{}

\newshadetheorem{xtheorem}{Theorem}[chapter]
\newshadetheorem{xlemma}[xtheorem]{Lemma}
\newshadetheorem{xdefinition}[xtheorem]{Definition}
\newshadetheorem{xrequirement}[xtheorem]{Requirement}

\newenvironment{theorem}[1][]{%
	\addtotheorems{Theorem}{#1}
	\begin{xtheorem}[#1]%
}{\end{xtheorem}}

\newenvironment{lemma}[1][]{%
	\addtotheorems{Lemma}{#1}
	\begin{xlemma}[#1]%
}{\end{xlemma}}

\newenvironment{definition}[1][]{%
	\addtotheorems{Definition}{#1}
	\begin{xdefinition}[#1]%
}{\end{xdefinition}}

\newenvironment{requirement}[1][]{%
	\addtotheorems{Requirement}{#1}
	\begin{xrequirement}[#1]%
}{\end{xrequirement}}

\newshadetheorem{xapxtheorem}{Theorem}[section]
\newenvironment{apxtheorem}[1][]{%
	\addtotheorems{Theorem}{#1}
	\begin{xapxtheorem}[#1]%
}{\end{xapxtheorem}}

\theoremstyle{nonumberplain}
\theoremseparator{.}
\theoremheaderfont{\sffamily\itshape}
\theoremsymbol{\ensuremath{\Box}}
\newtheorem{proof}{Proof}
\newtheorem{apxproof}{Proof}

% Listings
\usepackage{listings}
\lstset{
	basewidth={0.5em,0.45em},
	frame=lines,
	framerule=\lightrulewidth,
	captionpos=b,
	lineskip=-1pt,
%	float=hbt,
	xleftmargin=1cm,
	xrightmargin=1cm,
	aboveskip=0.5cm,
	belowskip=0.5cm,
}
\lstnewenvironment{java}[2]{
	\refstepcounter{listing}
	\addcontentsline{lst}{listing}{\protect\numberline{\thelisting}#1}
	\lstset{
		language=C++,
		basicstyle=\ttfamily,
		commentstyle=\sffamily,
		lineskip=-2pt,
		tabsize=4,
		#2
	}
}{}
\lstdefinelanguage{pseudocode}{
	sensitive=true,
	alsodigit={:},
	morekeywords={
		Algorithmus,Eingabe:,Ausgabe:,Variablen:,%
		when,if,then,else,end,atomic,out,%
		case,is,repeat,while,do,until},
}
\lstdefinelanguage[distributed]{pseudocode}
	[]{pseudocode}{
	alsodigit={:},
	morekeywords={Async,Sync,%
		Constants:,Variables:,Input:,Action:,Action,%
		send,to,from},
	deletekeywords={Algorithmus,Eingabe:,Ausgabe:,Variablen:}
}
\newcommand{\setpseudocode}[2]{
	\lstset{
		% Syntax
		language=[#1]{pseudocode},
		mathescape=true,
		escapechar=\#,
		tabsize=4,
%		gobble=4,
		literate={:=}{{$\gets$}{\:}}2 {->}{{$\rightarrow$}{\:}}2,
		% Style
		basicstyle=\rmfamily,
		commentstyle=\sffamily,
		moredelim=[is][\itshape]{//}{//},
		moredelim=[il][\sffamily]{**},
		% Layout & Placement
		numbers=left,
		numberstyle=\tiny,
		columns=flexible,
		breaklines=true,
		breakatwhitespace=true,
		breakindent=2em,
		% User defined
		#2
	}
}
\lstnewenvironment{distalg}[2]{%
	\refstepcounter{listing}
	\addcontentsline{lst}{listing}{\protect\numberline{\thelisting}#1}
	\setpseudocode{distributed}{#2}
}{}
\lstnewenvironment{pseudocode}[2]{%
	\refstepcounter{listing}
	\addcontentsline{lst}{listing}{\protect\numberline{\thelisting}#1}
	\setpseudocode{}{#2}
}{}

\newcommand\textcall[1]{\textsc{#1}}
\newcommand\call[2]{\ensuremath{\operatorname{\textcall{#1}}(#2)}}
\newcommand\callname[2]{\ensuremath{\operatorname{#1}(#2)}}
\newcommand\topin[1]{\ensuremath{\operatorname{In}({#1})}}
\newcommand\topout[1]{\ensuremath{\operatorname{Out}({#1})}}


%
% Inline
\usepackage{url}		% Cite URLs
\usepackage{numprint}		% Format numbers
\newcommand\notice[1]{}		% Notice
\newcommand\seppar{ \vspace{2ex} \noindent } % New paragraph
\newcommand\name[1]{{\em #1}} 	% Names


\newenvironment{BNF}
  {\captionsetup{type=lstlisting}}
  {}
  
\renewcommand{\labelitemi}{$\bullet$}
\renewcommand{\labelitemii}{$\bullet$}
\renewcommand{\labelitemiii}{$\bullet$}
\renewcommand{\labelitemiv}{$\bullet$}

\usepackage[inline]{enumitem}


\usepackage{xcolor}
\usepackage[tikz]{bclogo}
\usepackage[framemethod=tikz]{mdframed}
\usepackage{lipsum}
\usepackage[many]{tcolorbox}

\definecolor{bgblue}{RGB}{245,243,253}
\definecolor{ttblue}{RGB}{91,194,224}

\mdfdefinestyle{mystyle}{%
  rightline=true,
  innerleftmargin=10,
  innerrightmargin=10,
  outerlinewidth=1pt,
  topline=false,
  rightline=true,
  bottomline=false,
  skipabove=\topsep,
  skipbelow=\topsep
}

\begin{document}

%
% Titelseite
\begin{titlepage}
\newlength{\detailwidth}
\newlength{\detaildescriptionwidth}
\settowidth{\detailwidth}{Eingereicht:~~ Prof. Dr.-Ing. habil. Winfried E. Kühnhauser}
\settowidth{\detaildescriptionwidth}{Studies:~~} 
\begin{center}

\vfill {
	\includegraphics[width=6cm]{logo-tu-ilmenau.jpg}
	\vspace{8ex}
}

\begin{normalsize}Technische Universität Ilmenau \\
Fakultät für Informatik und Automatisierung \\
Institut für Praktische Informatik und Medieninformatik \\
Fachgebiet für Verteilte Systeme und Betriebssysteme        
\end{normalsize}


\vfill {\large
	Masters thesis\\ \vspace{2ex}
	\LARGE \textbf{Graphical Specification Language for the Entity-Labeling Aspect} \\ \vspace{2.5ex} 
	\normalsize Submitted by: \\
	\Large Philipp Schwetschenau \\ \vspace{3ex} 
}
\vfill \parbox{\detailwidth}{
	\makebox[\detaildescriptionwidth][r]{Supervisor:~~} Prof. Dr.-Ing. habil. Winfried E. Kühnhauser \\
	\makebox[\detaildescriptionwidth][r]{Supervisor:~~} Peter Amthor \\
	\vspace{2ex} \\
	\makebox[\detaildescriptionwidth][r]{Studies:~~} Computer science\\  
	\makebox[\detaildescriptionwidth][r]{Matriculation no.:~~} 46756  
	\vspace{2ex} \\
	\makebox[\detaildescriptionwidth][r]{Submission date:~~} Ilmenau, 29. November 2018 
}
\vspace*{\fill}
\end{center}
\end{titlepage}

\mbox{}
\thispagestyle{empty}
\cleardoublepage

% Gliederung
\pagenumbering{roman}
\setcounter{page}{1}
\tableofcontents

% Abbildungsverzeichnis
\cleardoublepage
\listoffigures
\tocentry{List of figures}

%\listoflisting
%\tocentry{\listingname}



% Inhalts-Teil
\cleardoublepage
\pagenumbering{arabic}
\chapter{Introduction}
\label{introduction}

\section{Domain} 
\label{domain}
With the increasing number of IT systems, securing these systems became an obvious and important issue.
For this purpose many security models and model families were developed for a wide field of application domains over the last years.

Formal security models offer possibilities to analyze them concerning security properties.
However, because quantity and variety of these models grow just as much as their relevance for security-critical applications the model-based security engineering process became more complex and therefore error-prone to human deviations.

\citet*{Amthor18} proposed a new approach called Aspect-oriented Security Engineering (AOSE) which claims to close semantic gaps between steps in the security engineering process (requirements, informal policy, formal model) to reduce the potential impact of human errors.
This approach roughly adopts the idea of the aspect-oriented programming paradigm.
It tailors all steps to aspects, which are non-functional requirements of the engineering process like determining requirements of policy semantics or analyzing certain security goals.
There are two major classes for aspects regarding AOSE: related to policy semantics and to policy analysis.

One possible aspect of the former is the Entity Labeling Aspect (EL).
It is designed to formally specify policy semantics typically found in operating systems and middleware systems.
Therefore it bridges the gap and supports the transformation between informal policy and formal model.
EL classifies model components into six semantic categories.
The notation is on a mathematical basis and uses concepts like sets, assignments or constraints.



\section{Motivation} 
\label{motivation}
The goal of reducing the impact of human errors by closing semantic gaps as much as possible with the help of AOSE/EL requires handling another formal notation, which is, in this case, EL itself and its classification into the six semantic categories.
To support working with this approach, especially for the communication between different groups of people involved like model engineers, security architects, software developers or future administrators, who have to cooperate and coordinate and all have different levels of experience, \citet*{Amthor18} considers a graphical representation to be helpful to enhance the transition between informal and formalized notation.

\citet*{Amthor18} already uses visual representations to illustrate examples.
However, the representations are not described in detail as well as they are inconsistent and ambiguous regarding several parts of the formalization (e.g.\ arrows can have several meanings and there is no way to specify functions with multiple input parameters).
So these visualizations are appropriate to underline and support the engineering of an already formalized policy after EL-based model engineering, but are not suitable for independent modeling on a stand-alone level, because they are not equivalent to their formalized mathematical counterpart.

There are two types of visualization \cite{Amthor18} uses, which have different levels of abstraction: 
One is based on the actual model components and visualizes them and their relationships.
The other one is based on a higher level of abstraction and visualizes the EL with its structure of semantic categories and their relationships.

There are other visual notations like UML or ERD, which have in common that they are tailored to particular needs in special application domains, so they can not be applied here, but may give inspiration on how to model certain semantics and relationships.

Eventually to be able to independently and visually model and work on EL-based policies there is need for an unambiguous formal graphical specification language.



\section{Goal}
\label{goal}
The main goal is to develop a graphical specification language for EL-based security polices.
This should focus on clean and unambiguous semantics of the language.

The use of the language should be as simple as possible and as comprehensive as needed.
Therefore its appearance and elements should be clear and well-structured.
A selection of appropriate symbols and geometrical shapes for their equivalent model counterparts has to be made regarding an intuitive understanding and workflow.
This selection should not be designed contradictory or conflicting to already established notations, especially those, which may be used in the context of security engineering.

It should be evaluated how feasible and reasonable it is to develop equivalent counterparts for every possible element and relationship in context of EL and to display all of them at once.
This may also lead to the question how the visualization is related to a visualization on a higher abstraction level and how they are connected and might be managed.
The latter may be investigated as an optional goal.

In addition to that a GUI-based editor should be developed as a prototype to make use of the proposed graphical specification language.



\section{Structure}
\label{structure}




\cleardoublepage
\chapter{Fundamentals}
\label{fundamentals}
This chapter provides information to all fundamentals necessary for this thesis.
It will serve as basis for all design decisions in chapter \ref{graphical_specification_language} and chapter \ref{editor}.

\section{Security models}
\label{security_models}


\subsection{Harrison-Ruzzo-Ullman}
\label{HRU}


\subsection{Role-based Access Control}
\label{RBAC}


\subsection{Security Model Core}
\label{model_core}




\section{Aspect-oriented Security Engineering and Entity-Labeling Aspect}
\label{AOSEEL}


\subsection{Aspect-oriented Security Engineering}
\label{AOSE}


\subsection{Entity-Labeling Aspect}
\label{EL}




\section{Graphical notation models ()}
\label{graphical_notations}

\subsection{Unified Modeling Language}
\label{UML}
The Unified Modeling Language (UML)

\subsection{Entity-Relationship}
\label{ER}
Entity-Relationship (ER)

\subsection{RBAC notation by Sandhu}
\label{RBAC_notation}
Sandhu useda notation to visualize his proposal to roled-based access control

maybe my own gsl from bachelor thesis?



\section{Gestalt laws and human optical perception}
\label{gestalt_laws_and_human_optical_perception}


\subsection{Gestalt laws}
\label{gestalt_laws}


\subsection{Human optical perception}
\label{human_optical_perception}




\section{GUI Design}
\label{gui_design}


\subsection{Design Patterns}
\label{design_patterns}


\subsection{Usability}
\label{usability}




\cleardoublepage
\chapter{Design: Graphical specification language}
\label{graphical_specification_language}
In diesem Kapitel sollen das Konzept und die notwendigen Elemente für eine graphisch-visuelle Repräsentation von Sicherheitsmodellen unter Beachtung der im Kapitel \ref{K2} vorgestellten Grundlagen erarbeitet werden. Ausgangspunkt bzw. Voraussetzung stellen hier Sicherheitsmodelle in Form des \textit{Security Model Core} dar. Wie dieses nach \cite{Poelck14} ...

\section{Concept (approach, basic ideas, adoptions from literature}

\section{Elements}

\section{Relationships}

\section{Structure}

\section{(optional) Visualization on the higher abstraction level}




\cleardoublepage
\chapter{Design: Editor GUI}
\label{editor}


\section{Structure}


\section{Sections}




\cleardoublepage
\chapter{Implementation}
\label{implementation}

	
\section{Implementation base (Qt, MVC))}


\section{Structure}

	
\section{GUI sections}




\cleardoublepage
\chapter{Evaluation}
\label{evaluation}


\section{Graphical specification language}


\section{Editor}




\cleardoublepage
\chapter{Conclusion}
\label{conclusion}




\cleardoublepage
\chapter{Summary}
\label{summary}




% Anhang
%\cleardoublepage
%\tocentry{Anhang}
%\renewcommand{\thesection}{\Alph{section}}
%\markboth{Anhang}{}
%\chapter*{Anhang}
%\section{Kapitel 1}
%\section{Kapitel 2}
%\label{appendix}




% Literaturverzeichnis
\cleardoublepage
\DeclareRobustCommand{\citeext}[1]{\citeauthor{#1}~\cite{#1}}
\bibliographystyle{plainnat}
\begin{thebibliography}{99}

\bibitem[Amthor(2018)]{Amthor18}
  Peter Amthor,
 \emph{An Aspect-oriented Approach to Model-based Security Engineering},
  Dissertation,
  Technische Universität Ilmenau,
  2018 

\bibitem[Sandhu(1996)]{Sandhu96}
 Ravi S. Sandhu, Edward J. Coyne, Hal L. Feinstein, Charles E. Youman,
 \emph{Role-Based Access Control Models},
 IEEE Computer, 
 29(2):38–47, 
 February 1996

\end{thebibliography}
\tocentry{Bibliography}

\cleardoublepage
%\input{other/selbststaendigkeit} %nicht notwendig für Studienarbeiten

\end{document}